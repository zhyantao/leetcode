\documentclass[a4paper]{ctexart}
\usepackage[landscape]{geometry}
\usepackage{multicol}
\usepackage{tikz}
\usepackage{listings}

\makeatletter

\advance\topmargin-.8in
\advance\textheight3in
\advance\textwidth3in
\advance\oddsidemargin-1.5in
\advance\evensidemargin-1.5in
\parindent0pt
\parskip2pt

\begin{document}

\begin{multicols*}{3}

  \tikzstyle{mybox} = [draw=black, fill=white, very thick,
  rectangle, rounded corners, inner sep=10pt, inner ysep=10pt]
  \tikzstyle{fancytitle} =[fill={rgb:red,220;green,220;blue,221},
  text=white, font=\bfseries]
  %------------ START OF CONTENT -------------


  %------------- START SECTION ---------------
  \begin{tikzpicture}
    \node [mybox] (box){%
      \begin{minipage}{0.3\textwidth}

        编码规范:\verb!*! 和 \verb!&! 紧挨变量名且与关键字之间保留一个空格。
        引用传递 \verb|int &b = a| 修改 \verb|b| 等于修改 \verb|a|。
        地址传递 \verb|int *b = &a| 修改 \verb|*b| 等于修改 \verb|a|。
        指针 \verb!*! 的优先级低于 \verb![]!,故 \verb!int *nums[3]! 是指针数组。\\
        \verb![捕获变量](参数列表) -> 返回类型 {函数主体}!
        是 Labmda 表达式,即匿名函数,其返回类型可以省略。

      \end{minipage}
    };
    %-----------------------------------------
    \node[fancytitle, right=10pt] at (box.north west) {指针和引用};
  \end{tikzpicture}


  %------------- START SECTION ---------------
  \begin{tikzpicture}
    \node [mybox] (box){%
      \begin{minipage}{0.3\textwidth}

        \begin{verbatim}
char arr[] = {'A', 0}; // 必须有 0 终止符
int arr[10]{ 0 }; // 必须赋初值 0,否则乱码
char *arr[] = {"ABC", "DEF"}; // 指针数组
\end{verbatim}

      \end{minipage}
    };
    %-----------------------------------------
    \node[fancytitle, right=10pt] at (box.north west) {字符数组的初始化};
  \end{tikzpicture}


  %------------- START SECTION ---------------
  \begin{tikzpicture}
    \node [mybox] (box){%
      \begin{minipage}{0.3\textwidth}

        \begin{verbatim}
char arr[] = "hello";
arr[0] = 'X'; // 正确,可以修改栈区
char *ptr = "world";
ptr[0] = 'X'; // 错误,不能修改常量区
\end{verbatim}

      \end{minipage}
    };
    %-----------------------------------------
    \node[fancytitle, right=10pt] at (box.north west) {字符数组的修改};
  \end{tikzpicture}


  %------------- START SECTION ---------------
  \begin{tikzpicture}
    \node [mybox] (box){%
      \begin{minipage}{0.3\textwidth}

        \begin{verbatim}
// 将数组复制到栈区
char str[] = "hello";
char buf[10];              // 开辟栈区空间
int sz = sizeof(str) / sizeof(str[0]);
//strcpy(buf, str);        // UNIX 习惯
strcpy_s(buf, sz, str);    // 不能用 b = a
strcmp(buf, str) == 0;     // 不能用 b == a

// 将数组复制到堆区
size_t len = strlen(str);
char *p = nullptr;
p = (char *) malloc(sizeof(char) * (len + 1));
strcpy_s(buf, len + 1, str);// 不能忘记 + 1
if (strcmp(buf, str) == 0)  // 不能用 b == a
  free(p);              // 释放动态开辟的空间
p = nullptr;                // 避免出现野指针
\end{verbatim}

      \end{minipage}
    };
    %-----------------------------------------
    \node[fancytitle, right=10pt] at (box.north west) {字符数组的复制};
  \end{tikzpicture}


  %------------- START SECTION ---------------
  \begin{tikzpicture}
    \node [mybox] (box){%
      \begin{minipage}{0.3\textwidth}

        \begin{verbatim}
Object *ptr = new Object[100];
delete[] ptr; // 删除对象数组,防止内存泄漏
char **ptr = new char *[100];
delete[] ptr; // 动态创建指针(字符串)数组
\end{verbatim}

      \end{minipage}
    };
    %-----------------------------------------
    \node[fancytitle, right=10pt] at (box.north west) {动态开辟空间 \verb!new / delete!};
  \end{tikzpicture}


  %------------- START SECTION ---------------
  \begin{tikzpicture}
    \node [mybox] (box){%
      \begin{minipage}{0.3\textwidth}

        \begin{verbatim}
ptr = (char *) str.c_str();// string -> char*
double d = atof("0.23");  // string -> double
int i = atoi("1021");     // string -> int
long l = atol("303992");  // string -> long
sprintf(chs, "%f", 2.3);  // 保存到 chs
char *ptr = chs;          // char[] -> char*
strcpy(chs, ptr, len);    // char* -> char[]
string str = chs;         // char[] -> string
\end{verbatim}

      \end{minipage}
    };
    %-----------------------------------------
    \node[fancytitle, right=10pt] at (box.north west) {类型转换 \verb!char chs[100]{ 0 }!};
  \end{tikzpicture}


  %------------- START SECTION ---------------
  \begin{tikzpicture}
    \node [mybox] (box){%
      \begin{minipage}{0.3\textwidth}

        \begin{verbatim}
vector<int> arr(sz, val); // sz, val 均可缺省 
vector<vector<int>> dp(m, vector<int>(n));
empty(), size()
push_back(), pop_back(), dp[i][j] = 13
\end{verbatim}

      \end{minipage}
    };
    %-----------------------------------------
    \node[fancytitle, right=10pt] at (box.north west) {动态数组 \verb!vector!};
  \end{tikzpicture}


  %------------- START SECTION ---------------
  \begin{tikzpicture}
    \node [mybox] (box){%
      \begin{minipage}{0.3\textwidth}

        \begin{verbatim}
string str = "ABCDEFG";
size(), empty()
push_back(), pop_back(), str[i] = 'X'
s1 == s2, substr(start, len)
\end{verbatim}

      \end{minipage}
    };
    %-----------------------------------------
    \node[fancytitle, right=10pt] at (box.north west) {字符串 \verb!string!};
  \end{tikzpicture}


  %------------- START SECTION ---------------
  \begin{tikzpicture}
    \node [mybox] (box){%
      \begin{minipage}{0.3\textwidth}

        \begin{verbatim}
unordered_map<string, int> map;
size(), empty(), count(key), emplace(k, v)
insert({key, val}), erase(key), at(key) = val
\end{verbatim}

      \end{minipage}
    };
    %-----------------------------------------
    \node[fancytitle, right=10pt] at (box.north west) {哈希表 \verb!unordered_map!};
  \end{tikzpicture}


  %------------- START SECTION ---------------
  \begin{tikzpicture}
    \node [mybox] (box){%
      \begin{minipage}{0.3\textwidth}

        \begin{verbatim}
unordered_set<string> set;
size(), empty(), count(key)
insert(key), erase(key)
\end{verbatim}

      \end{minipage}
    };
    %-----------------------------------------
    \node[fancytitle, right=10pt] at (box.north west) {哈希集合 \verb!unordered_set!};
  \end{tikzpicture}


  %------------- START SECTION ---------------
  \begin{tikzpicture}
    \node [mybox] (box){%
      \begin{minipage}{0.3\textwidth}

        \begin{verbatim}
queue<sting> q;
size(), empty(), push(), pop(), front()
\end{verbatim}

      \end{minipage}
    };
    %-----------------------------------------
    \node[fancytitle, right=10pt] at (box.north west) {队列 \verb!queue!};
  \end{tikzpicture}


  %------------- START SECTION ---------------
  \begin{tikzpicture}
    \node [mybox] (box){%
      \begin{minipage}{0.3\textwidth}

        \begin{verbatim}
stack<string> s;
size(), empty(), push(), pop(), top()
\end{verbatim}

      \end{minipage}
    };
    %-----------------------------------------
    \node[fancytitle, right=10pt] at (box.north west) {堆栈 \verb!stack!};
  \end{tikzpicture}


  %------------- START SECTION ---------------
  \begin{tikzpicture}
    \node [mybox] (box){%
      \begin{minipage}{0.3\textwidth}

        作为类成员时,重载二元运算符参数只有一个,重载一元运算符不需要参数。
        作为全局函数时,重载二元运算符需要两个参数,重载一元运算符需要一个参数。

        \begin{verbatim}
// 作为成员函数
Complex Complex::operator+(Complex &a) {
  Complex b; // 复数
  b.real = this->real + a.real; // 实部
  b.img = this->img + a.img; // 虚部
  return b;
}
// 作为全局函数
Complex operator+(Complex &a, int b) {
  return Complex(a.real + b, a.img);
}
\end{verbatim}

        将运算符重载函数作为全局函数时,一般都需要在类中将该函数声明为友元函数。

      \end{minipage}
    };
    %-----------------------------------------
    \node[fancytitle, right=10pt] at (box.north west) {运算符重载};
  \end{tikzpicture}


  %------------- START SECTION ---------------
  \begin{tikzpicture}
    \node [mybox] (box){%
      \begin{minipage}{0.3\textwidth}

        \begin{verbatim}
sort(v1.begin(), v1.end(), greater<int>());
sort(v1.begin(), v1.end(), [](int a, int b) {
  return a > b; // 降序排列
});
reverse(v1.begin(), v1.end());
binary_search(v1.begin(), v1.end(), target);
\end{verbatim}

      \end{minipage}
    };
    %-----------------------------------------
    \node[fancytitle, right=10pt] at (box.north west) {工具函数};
  \end{tikzpicture}


  %------------ END OF CONTENT ---------------
\end{multicols*}

\end{document}
