\documentclass[a4paper]{ctexart}
\usepackage[landscape]{geometry}
\usepackage{multicol}
\usepackage{tikz}
\usepackage{listings}

\makeatletter

\advance\topmargin-.8in
\advance\textheight3in
\advance\textwidth3in
\advance\oddsidemargin-1.5in
\advance\evensidemargin-1.5in
\parindent0pt
\parskip2pt

\begin{document}

\begin{multicols*}{3}

  \tikzstyle{mybox} = [draw=black, fill=white, very thick,
  rectangle, rounded corners, inner sep=10pt, inner ysep=10pt]
  \tikzstyle{fancytitle} =[fill={rgb:red,220;green,220;blue,221},
  text=white, font=\bfseries]
  %------------ START OF CONTENT -------------


  %------------- START SECTION ---------------
  \begin{tikzpicture}
    \node [mybox] (box){%
      \begin{minipage}{0.3\textwidth}

        Java 引用传递 \verb|ArrayList<Integer> b = a| 可以修改自定义类型的值。
        Java 无地址传递,在递归遍历树的同时想修改变量,\textbf{声明为全局变量}可降低难度。

      \end{minipage}
    };
    %-----------------------------------------
    \node[fancytitle, right=10pt] at (box.north west) {指针和引用};
  \end{tikzpicture}



  %------------- START SECTION ---------------
  \begin{tikzpicture}
    \node [mybox] (box){%
      \begin{minipage}{0.3\textwidth}

        初始化:
        \verb!int[] v = new int[];!\\
        \verb!boolean[][] vv = new boolean[m][n];!\\
        判空:
        \verb!v == null || v.length == 0 // 一维!\\
        \verb!vv == null || vv.length == null || ! \\
        \verb!              vv[0].length == 0 // 二维!\\
        截取:
        \verb!subarr = Arrays.copyOfRange(arr,2,6);!

      \end{minipage}
    };
    %-----------------------------------------
    \node[fancytitle, right=10pt] at (box.north west) {数组 int [ ]};
  \end{tikzpicture}



  %------------- START SECTION ---------------
  \begin{tikzpicture}
    \node [mybox] (box){%
      \begin{minipage}{0.3\textwidth}

        初始化:
        \verb!String str = "hello world";!\\
        \verb!StringBuilder sb = new StringBuilder(str);!\\
        类型转换:\verb!sb.toString();!\\
        增:
        \verb!sb.append(true);  sb.insert(i, "abc");!\\
        删:
        \verb!sb.deleteCharAt(i);    sb.delete(i,j);!\\
        改:
        \verb!sb.setCharAt(i,'z');!\\
        查:
        \verb!char c = str.charAt(i);!\\
        判等:
        \verb!s1.equals(s2);!
        判空:
        \verb!s1.isEmpty();!\\
        截取:
        \verb!str.substring(i); str.substring(i,j);!\\
        拼接:
        \verb!str.concat("abc");!

      \end{minipage}
    };
    %-----------------------------------------
    \node[fancytitle, right=10pt] at (box.north west) {字符串 String};
  \end{tikzpicture}



  %------------- START SECTION ---------------
  \begin{tikzpicture}
    \node [mybox] (box){%
      \begin{minipage}{0.3\textwidth}

        \verb!ArrayList<Integer> v = new ArrayList<>();!\\
        \verb!isEmpty(), size()!\\
        \verb!add(), remove(), get(), set(1, 100)!

      \end{minipage}
    };
    %-----------------------------------------
    \node[fancytitle, right=10pt] at (box.north west) {动态数组};
  \end{tikzpicture}



  %------------- START SECTION ---------------
  \begin{tikzpicture}
    \node [mybox] (box){%
      \begin{minipage}{0.3\textwidth}

        \verb!LinkedList<Integer> v = new LinkedList<>();!\\
        \verb!isEmpty(), size(), contains()!\\
        \verb!add(), remove(), get() // 默认尾插,头删!\\
        \verb!addFirst(), removeFirst(), removeLast()!

      \end{minipage}
    };
    %-----------------------------------------
    \node[fancytitle, right=10pt] at (box.north west) {双向链表};
  \end{tikzpicture}

  \hfill



  %------------- START SECTION ---------------
  \begin{tikzpicture}
    \node [mybox] (box){%
      \begin{minipage}{0.3\textwidth}

        \verb!Map<Integer, String> map = new HashMap<>();!\\
        \verb!containsKey(), keySet(), getOrDefault()!\\
        \verb!put(), remove(), get()!

      \end{minipage}
    };
    %-----------------------------------------
    \node[fancytitle, right=10pt] at (box.north west) {哈希表(哈希集合类似)};
  \end{tikzpicture}



  %------------- START SECTION ---------------
  \begin{tikzpicture}
    \node [mybox] (box){%
      \begin{minipage}{0.3\textwidth}

        \verb!Map<String> map = new TreeSet<>();!\\
        \verb!put(), remove(), containsKey(), firstKey()!\\
        \verb!lastKey(), floorKey(), ceilingKey()!

      \end{minipage}
    };
    %-----------------------------------------
    \node[fancytitle, right=10pt] at (box.north west) {有序表(有序集合类似)};
  \end{tikzpicture}



  %------------- START SECTION ---------------
  \begin{tikzpicture}
    \node [mybox] (box){%
      \begin{minipage}{0.3\textwidth}

        \verb!Deque<String> stack = new ArrayDeque<>();!\\
        \verb!Deque<String> queue = new LinkedDeque<>();!\\
        \verb!isEmpty(), size()!\\
        \verb!addFirst(), removeFirst(), peekFirst()!\\
        \verb!addLast(), removeLast(), peekLast()!

      \end{minipage}
    };
    %-----------------------------------------
    \node[fancytitle, right=10pt] at (box.north west) {双端队列(代替栈和队列)};
  \end{tikzpicture}



  %------------- START SECTION ---------------
  \begin{tikzpicture}
    \node [mybox] (box){%
      \begin{minipage}{0.3\textwidth}

        % 代码段要顶格写,否则,会看起来跑到 minipage 外面去了
        \begin{verbatim}
Integer.parseInt(s) // String -> int
String.valueOf(chs) // int, char[] -> String
'8' - '0'           // char -> int
Double.valueOf(i)   // int -> double
foo.intValue()      // double -> int
list = Arrays.asList(arr) // [] -> ArrayList 
\end{verbatim}

      \end{minipage}
    };
    %-----------------------------------------
    \node[fancytitle, right=10pt] at (box.north west) {类型转换};
  \end{tikzpicture}



  %------------- START SECTION ---------------
  \begin{tikzpicture}
    \node [mybox] (box){%
      \begin{minipage}{0.3\textwidth}

        % 代码段要顶格写,否则,会看起来跑到 minipage 外面去了
        \begin{verbatim}
TreeNode p = new TreeNode(-1, head);
TreeNode dummy = p; // 保存头结点位置,不移动
return dummy.next;
\end{verbatim}

      \end{minipage}
    };
    %-----------------------------------------
    \node[fancytitle, right=10pt] at (box.north west) {哨兵:哑元结点(提高效率)};
  \end{tikzpicture}



  %------------- START SECTION ---------------
  \begin{tikzpicture}
    \node [mybox] (box){%
      \begin{minipage}{0.3\textwidth}

        % 代码段要顶格写,否则,会看起来跑到 minipage 外面去了
        \begin{verbatim}
Arrays.sort(nums);       // 数组排序
Arrays.binarySearch(nums, 23);
Arrays.stream(nums).max().getAsInt();
Collections.sort(list); // 列表排序
list.sort(Collections.reverseOrder());// 逆序
Collections.reverse(list); // 翻转链表
\end{verbatim}

      \end{minipage}
    };
    %-----------------------------------------
    \node[fancytitle, right=10pt] at (box.north west) {Arrays 和 Collections 工具包};
  \end{tikzpicture}



  %------------- START SECTION ---------------
  \begin{tikzpicture}
    \node [mybox] (box){%
      \begin{minipage}{0.3\textwidth}

        默认 int 无符号最大值约为 $40 \times 10^8$(32 位)。\\
        \verb!BigInteger A = BigInteger.valueOf(23);!\\
        \verb!BigDecimal B = BigDecimal.valueOf(1234.56);!\\
        \verb!A.add(A), A.subtract(A), !\\
        \verb!A.multiply(A), A.divide(A)!

      \end{minipage}
    };
    %-----------------------------------------
    \node[fancytitle, right=10pt] at (box.north west) {大数计算:java.math.*};
  \end{tikzpicture}



  %------------- START SECTION ---------------
  \begin{tikzpicture}
    \node [mybox] (box){%
      \begin{minipage}{0.3\textwidth}

        基于数组实现:\verb!key! 为数组下标,\verb!value! 为元素值。\\
        \verb!int[] map = new int[256]; // ASCII -> 下标!

      \end{minipage}
    };
    %-----------------------------------------
    \node[fancytitle, right=10pt] at (box.north west) {简易哈希表的实现};
  \end{tikzpicture}



  %------------- START SECTION ---------------
  \begin{tikzpicture}
    \node [mybox] (box){%
      \begin{minipage}{0.3\textwidth}

        % 代码段要顶格写,否则,会看起来跑到 minipage 外面去了
        \begin{verbatim}
ArrayList<String> list = new ArrayList<>();
Set<String> set = new HashSet<>();
set.addAll(list);
ArrayList<String> ret = new ArrayList<>(set);
\end{verbatim}

      \end{minipage}
    };
    %-----------------------------------------
    \node[fancytitle, right=10pt] at (box.north west) {ArrayList 元素去重};
  \end{tikzpicture}



  %------------- START SECTION ---------------
  \begin{tikzpicture}
    \node [mybox] (box){%
      \begin{minipage}{0.3\textwidth}

        % 代码段要顶格写,否则,会看起来跑到 minipage 外面去了
        \begin{verbatim}
while (n > 0) {
  digit = n % 10;   // 取出各位
  n = n / 10;       // 更新数字
} // 注意处理最值 Integer.MAX_VALUE
\end{verbatim}

      \end{minipage}
    };
    %-----------------------------------------
    \node[fancytitle, right=10pt] at (box.north west) {取出整数中的每一位};
  \end{tikzpicture}



  %------------- START SECTION ---------------
  \begin{tikzpicture}
    \node [mybox] (box){%
      \begin{minipage}{0.3\textwidth}

        \begin{verbatim}
Map<String, String> map = new HashMap<>();
for (Map.Entry<String, String> entry : 
                          map.entrySet()) {
  int key = entry.getKey();
  int value = entry.getValue();
}
\end{verbatim}

      \end{minipage}
    };
    %-----------------------------------------
    \node[fancytitle, right=10pt] at (box.north west) {遍历哈希表或哈希集合};
  \end{tikzpicture}



  %------------- START SECTION ---------------
  \begin{tikzpicture}
    \node [mybox] (box){%
      \begin{minipage}{0.3\textwidth}

        % 代码段要顶格写,否则,会看起来跑到 minipage 外面去了
        \begin{verbatim}
void traverse(ListNode head) {
  // 前序遍历代码
  traverse(head.next);
  // 后序遍历代码
}
\end{verbatim}

      \end{minipage}
    };
    %-----------------------------------------
    \node[fancytitle, right=10pt] at (box.north west) {链表的遍历};
  \end{tikzpicture}



  %------------- START SECTION ---------------
  \begin{tikzpicture}
    \node [mybox] (box){%
      \begin{minipage}{0.3\textwidth}

        % 代码段要顶格写,否则,会看起来跑到 minipage 外面去了
        \begin{verbatim}
void traverse(TreeNode root) {
  if (root == null) return;
  // 前序遍历代码
  traverse(root.left);
  // 中序遍历代码
  traverse(root.right);
  // 后序遍历代码
}
\end{verbatim}

      \end{minipage}
    };
    %-----------------------------------------
    \node[fancytitle, right=10pt] at (box.north west) {二叉树的遍历};
  \end{tikzpicture}



  %------------- START SECTION ---------------
  \begin{tikzpicture}
    \node [mybox] (box){%
      \begin{minipage}{0.3\textwidth}

        % 代码段要顶格写,否则,会看起来跑到 minipage 外面去了
        \begin{verbatim}
class TreeNode {
  int val;
  TreeNode[] children;
}
void traverse(TreeNode root) {
  for (TreeNode child : children)
    traverse(child);
}
\end{verbatim}

      \end{minipage}
    };
    %-----------------------------------------
    \node[fancytitle, right=10pt] at (box.north west) {图(N 叉树)的遍历};
  \end{tikzpicture}



  %------------- START SECTION ---------------
  \begin{tikzpicture}
    \node [mybox] (box){%
      \begin{minipage}{0.3\textwidth}

        \begin{verbatim}
// 利用这个例子学习如何使用递归的返回值
int traverse(TreeNode root) {
  // 叶节点相当于 dp 的最后一个状态
  if (root == null)
    return 0; // 实际意义: 0 层
  // 从叶结点到根结点逆向推理 left 和 right
  int left = traverse(root.left);
  int right = traverse(root.right);
  // 下一次递归利用上一次递归 return 的结果
  return left>right ? left+1 : right+1;
}
\end{verbatim}

      \end{minipage}
    };
    %-----------------------------------------
    \node[fancytitle, right=10pt] at (box.north west) {求二叉树的深度:递归实现};
  \end{tikzpicture}



  %------------- START SECTION ---------------
  \begin{tikzpicture}
    \node [mybox] (box){%
      \begin{minipage}{0.3\textwidth}

        % 代码段要顶格写,否则,会看起来跑到 minipage 外面去了
        \begin{verbatim}
Queue<Integer> queue = new LinkedList<>();
void traverse(TreeNode root) {
  if (root != null)
    queue.offer(root);
  else
    return;

  TreeNode p = queue.poll();
  while (p != null) {
    if (p.left != null)
      queue.offer(p.left);
    if (p.right != null)
      queue.offer(p.right);
    if ( ! queue.isEmpty())
      p = queue.poll();
  }
}
\end{verbatim}

      \end{minipage}
    };
    %-----------------------------------------
    \node[fancytitle, right=10pt] at (box.north west) {二叉树的层序遍历};
  \end{tikzpicture}



  %------------- START SECTION ---------------
  \begin{tikzpicture}
    \node [mybox] (box){%
      \begin{minipage}{0.3\textwidth}

        \begin{verbatim}
// 类名必须为 Main,不含 package xxx 信息
public class Main {
  public static void main(String[] args) {
    Scanner in = new Scanner(System.in);
    // 若有下一个字符 hasNext 返回真
    // 若碰到行尾符号 hasNextLine 返回真
    // 注意 hasNextXXX 与 nextXXX 须同时出现
    while (in.hasNextInt()) { // 检查
      int a = in.nextInt();
      int b = in.nextInt(); // 指针向前移动
      // 四舍五入,保留两位小数
      String.format("%.2f", num);
    }
  }
}
\end{verbatim}

      \end{minipage}
    };
    %-----------------------------------------
    \node[fancytitle, right=10pt] at (box.north west) {ACM 模式:import java.util.Scanner};
  \end{tikzpicture}



  %------------- START SECTION ---------------
  \begin{tikzpicture}
    \node [mybox] (box){%
      \begin{minipage}{0.3\textwidth}

        \begin{verbatim}
// 以补码形式存储,以原码形式输出到屏幕
// int 类型按道理应该写 32 位,下面写法不严谨
9     // 原码 01001 补码 01001 反码 01001
-1    // 原码 10001 补码 11111 反码 11110
~ 9   // ~ 01001 = 10110 原码 11010 = -10
a ^ a    // = 0,判断两数是否相同
a ^ 0x1  // 对 a 的第 0 位求反
a ^ b    // 求 a + b 的各位之和,无进位
a & b    // 求 a + b 各位的进位
a & 0xFE // 关闭或检查 a 的第 0 位
a | 0x01 // 开启 a 的第 0 位
n & (~ n + 1) // 获取 n 的二进制最右侧的 1
n &= (n - 1)  // 抹掉 n 的二进制最后侧的 1

int add(int a,int b) {
  int sum = a; // 求两数之和的例子
  int add = b;
  while (add != 0) {
    int tmp = sum ^ add;
    add = (sum & add) << 1;
    sum = tmp;
  }
  return sum;
}
\end{verbatim}

        若出现新的运算规则,题目实际想让自己定义运算。

      \end{minipage}
    };
    %-----------------------------------------
    \node[fancytitle, right=10pt] at (box.north west) {位运算};
  \end{tikzpicture}



  %------------- START SECTION ---------------
  \begin{tikzpicture}
    \node [mybox] (box){%
      \begin{minipage}{0.3\textwidth}

        $T(N) = a * T(\displaystyle\frac{N}{b}) + O(N^d)$

        $N$ 是问题的总规模,
        $a$ 是递归调用的次数,
        $\frac{N}{b}$ 是子问题的规模,
        $O(N^d)$ 是除递归代码外的时间复杂度。

        - $log_b a < d \Rightarrow O(N^d)$ \\
        - $log_b a > d \Rightarrow O(Nlog_b a)$ \\
        - $log_b a = d \Rightarrow O(N^d logN)$

      \end{minipage}
    };
    %-----------------------------------------
    \node[fancytitle, right=10pt] at (box.north west) {计算递归的时间复杂度:master 公式};
  \end{tikzpicture}



  %------------- START SECTION ---------------
  \begin{tikzpicture}
    \node [mybox] (box){%
      \begin{minipage}{0.3\textwidth}

        \begin{verbatim}
// 注意这里有多处使用 return
int binarySearch(int[] arr, int target,
                 int left, int right) {
  if (left <= right) {
    int mid = left + (right - left) / 2;
    if (arr[mid] == target)
      return mid;
    if (arr[mid] > target) // 向左查找
      return binarySearch(arr, target,
                          left, mid - 1);
    if (arr[mid] < target) // 向右查找
      return binarySearch(arr, target,
                          mid + 1, right);
  }
  return -1; // 没找到
}
\end{verbatim}

      \end{minipage}
    };
    %-----------------------------------------
    \node[fancytitle, right=10pt] at (box.north west) {二分查找:递归实现};
  \end{tikzpicture}



  %------------- START SECTION ---------------
  \begin{tikzpicture}
    \node [mybox] (box){%
      \begin{minipage}{0.3\textwidth}

        \begin{verbatim}
void shellSort(int[] arr) {
  for (int step = arr.length / 2;
           step >= 1; step /= 2) {
    for (int r = step; r < len; r++) {
      int tmp = arr[r]; // 把 r 放到最终位置
      int l = r - step;
      while (l >= 0 && arr[l] > tmp) {
        arr[l + step] = arr[l]; // 将 l 右移
        l -= step;
      }
      arr[l + step] = tmp; // 放置
    }
  }
}
\end{verbatim}

      \end{minipage}
    };
    %-----------------------------------------
    \node[fancytitle, right=10pt] at (box.north west) {希尔排序:从小到大};
  \end{tikzpicture}



  %------------- START SECTION ---------------
  \begin{tikzpicture}
    \node [mybox] (box){%
      \begin{minipage}{0.3\textwidth}

        % 代码段要顶格写,否则,会看起来跑到 minipage 外面去了
        \begin{verbatim}
// 向已有堆的末尾插入元素,重建大根堆
void heapInsert(int[] arr, int i) {
  while (arr[i] > arr[(i - 1) / 2]) {
    swap(arr, i, (i - 1) / 2);
    i = (i - 1) / 2;
  }
}
// 移除堆顶元素(放在末尾),重建大根堆
heapify(int[] arr, int i, int heapSize) {
  while (i < heapSize) {
    int l = 2 * i + 1; // 左孩子指针
    int r = 2 * i + 2; // 右孩子指针
    int max = i;
    if (l < heapSize && arr[l] > arr[max])
      max = l;
    if (r < heapSize && arr[r] > arr[max])
      max = r;
    if (max == i)
      break;
    swap(arr, i, max);
    i = max;
  }
}
\end{verbatim}

      \end{minipage}
    };
    %-----------------------------------------
    \node[fancytitle, right=10pt] at (box.north west) {重建大根堆:树形数组、完全二叉树、优先队列};
  \end{tikzpicture}



  %------------- START SECTION ---------------
  \begin{tikzpicture}
    \node [mybox] (box){%
      \begin{minipage}{0.3\textwidth}

        \begin{verbatim}
void insertSort(int[] arr) {
  int j; // 用于扫描 i 之前的元素
  for (int i = 1; i < arr.length; i++) {
    int tmp = arr[i];
    for (j = i; j > 0 && arr[j-1] > tmp; j--)
      arr[j] = arr[j-1]; // 向后移动元素
    arr[j] = tmp;
  }
}
\end{verbatim}

      \end{minipage}
    };
    %-----------------------------------------
    \node[fancytitle, right=10pt] at (box.north west) {插入排序 $O(n^2)$:从小到大};
  \end{tikzpicture}



  %------------- START SECTION ---------------
  \begin{tikzpicture}
    \node [mybox] (box){%
      \begin{minipage}{0.3\textwidth}

        PriorityQueue 就是一个小根堆结构,可以直接使用。

        % 代码段要顶格写,否则,会看起来跑到 minipage 外面去了
        \begin{verbatim}
heapSort(int[] arr) {
  if (arr == null || arr.length < 2)
    return;
  // 构建大根堆(方法一)
  //for (int i = 0; i < arr.length; i++)
  //  heapInsert(arr, i);
  // 构建大根堆(方法二,更快)
  for (int i = arr.length - 1; i >= 0; i--)
    heapify(arr, i, arr.length);
  // 每次选择并移除堆顶元素,放到末尾
  int heapSize = arr.length;
  swap(arr, 0, --heapSize);
  while (heapSize > 0) {
    heapify(arr, 0, heapSize);
    swap(arr, 0, --heapSize);
  }
}
\end{verbatim}

      \end{minipage}
    };
    %-----------------------------------------
    \node[fancytitle, right=10pt] at (box.north west) {堆排序:从小到大(利用 heapInsert 和 heapify)};
  \end{tikzpicture}



  %------------- START SECTION ---------------
  \begin{tikzpicture}
    \node [mybox] (box){%
      \begin{minipage}{0.3\textwidth}

        \begin{verbatim}
mergeSort(int[] arr, int[] tmp,
          int left, int right) {
  if (left < right) {
    int mid = left + (right - left) / 2;
    mergeSort(arr, tmp, left, mid);
    mergeSort(arr, tmp, mid+1, right);
    merge(arr, tmp, left, mid, right);
  }
}
merge(int[] arr, int[] tmp,
      int left, int mid, int right) {
  int pLeft = left
  int pRight = mid + 1;
  int pTmp = left;
  // 将左右子数组较小的元素依次插入到 tmp 中
  while (pLeft <= mid && pRight <= right) {
    if (arr[pLeft] <= arr[pRight])
      tmp[pTmp++] = arr[pLeft++];
    else
      tmp[pTmp++] = arr[pRight++];
  }
  // 复制剩余元素到 tmp 中
  while (pLeft <= mid)
    tmp[pTmp++] = arr[pLeft++];
  while (pRight <= right)
    tmp[pTmp++] = arr[pRight++];
  // 必须保存局部的排序结果,否则下次还是乱序
  for (int i = left; i <= right; i++)
    arr[i] = tmp[i];
}
调用:int[] arr = new int[]{7, 3, 2, 6};
int[] tmp = new int[arr.length]; // 辅助空间
mergeSort(arr, tmp, 0, arr.length - 1);
\end{verbatim}

      \end{minipage}
    };
    %-----------------------------------------
    \node[fancytitle, right=10pt] at (box.north west) {归并排序 $O(nlog(n))$:分而治之(递归)};
  \end{tikzpicture}



  %------------- START SECTION ---------------
  \begin{tikzpicture}
    \node [mybox] (box){%
      \begin{minipage}{0.3\textwidth}

        % 代码段要顶格写,否则,会看起来跑到 minipage 外面去了
        \begin{verbatim}
quickSort(int[] arr, int left, int right) {
  if (left < right) {
    int pivot = arr[left];   // 随机选基准点
    int i = left, j = right; // 不修改原变量
    while (i < j) {
      while (i < j && arr[j] > pivot)
        j--; // 从右往左:首个比 pivot 小的值
      if (i < j) {
        arr[i] = arr[j]; // 丢失 arr[i]
        i++;
      }
      while (i < j && arr[i] < pivot)
        i++; // 从左往右:首个比 pivot 大的值
      if (i < j) {
        arr[j] = arr[i];
        j--;
      }
    }
    arr[i] = pivot; // 找回 arr[i]
    // -- partition 和递归代码的分割线 -- //
    quickSort(arr, left, i - 1);
    quickSort(arr, i + 1, right);
  }
}
调用:quickSort(arr, 0, arr.length - 1);
\end{verbatim}

      \end{minipage}
    };
    %-----------------------------------------
    \node[fancytitle, right=10pt] at (box.north west) {快速排序 $O(nlog(n))$:从小到大};
  \end{tikzpicture}



  %------------- START SECTION ---------------
  \begin{tikzpicture}
    \node [mybox] (box){%
      \begin{minipage}{0.3\textwidth}

        \begin{verbatim}
// 单调队列,要始终维持队列递增或递减的状态。
// 递增(减)队列的队头是最小(大)值。
int[] maxSlidingWindow(int[] arr, int sz) {
    int[] ans = new int[arr.length - sz + 1];
    Deque<Integer> deque = new LinkedList<>();
    // r 表示滑动窗口右边界
    for (int r = 0; r < arr.length; r++) {
      // 移除队尾比当前值小的元素的索引
      while (!deque.isEmpty()
          && arr[r] >= arr[deque.peekLast()])
        deque.removeLast();
      deque.addLast(r);      // 存储元素下标
      int l = r - sz + 1;    // 窗口左边界
      if (deque.peekFirst() < l)//超出左边界
        deque.removeFirst();
      if (r + 1 >= sz) // 若已经形成窗口
        ans[l] = arr[deque.peekFirst()];
    }
    return ans;
}
\end{verbatim}

      \end{minipage}
    };
    %-----------------------------------------
    \node[fancytitle, right=10pt] at (box.north west) {求滑动窗口中的最大值:单调队列、双端队列};
  \end{tikzpicture}



  %------------- START SECTION ---------------
  \begin{tikzpicture}
    \node [mybox] (box){%
      \begin{minipage}{0.3\textwidth}

        核心思想:适用小问题的算法也适用于大问题\\
        1. 找出基线条件(递归终止条件)\\
        2. 不断将问题分解,直到符合基线条件\\
        涉及数组时基线条件通常是数组为空或只含一个元素

        % 代码段要顶格写,否则,会看起来跑到 minipage 外面去了
        \begin{verbatim}
// 求和:1 + ... + 50,// traverse(dp, 0)
int traverse(int[] dp, int i) {
  // 如果依赖未来的知识,需要知道最后状态的值
  if (i == dp.length)
    return 50;
  return i + traverse(dp, i + 1);
}
\end{verbatim}

      \end{minipage}
    };
    %-----------------------------------------
    \node[fancytitle, right=10pt] at (box.north west) {递归的基本思路};
  \end{tikzpicture}



  %------------- START SECTION ---------------
  \begin{tikzpicture}
    \node [mybox] (box){%
      \begin{minipage}{0.3\textwidth}

        先暴力递归,再使用优化技巧(消除重复计算)。\\
        1. 直接写出初始状态 \verb!dp[0]!, \verb!dp[1]! 的答案;\\
        2. 当前状态 \verb!dp[i]! 的通常是题目要求的结果;\\
        3. \verb!dp[i]! 与 \verb!dp[i-2:i+2]! 有何关联?\\
        4. \verb!dp[i,j]! 与 \verb!dp[i-2:i+2,j-2:j+2]! 有何关联? \\
        注:\verb!dp[i]! 是状态数组,表示第 \verb!i! 个状态,
        也就是数组中的第 \verb!i! 个元素。
        通常,整个数组作为\textbf{额外的不变参数}一直传递,用于判断是否访问越界。

        % 代码段要顶格写,否则,会看起来跑到 minipage 外面去了
        \begin{verbatim}
// 1. 暴力递归求斐波那契数列:0 1 1 2 3 ...
int fib(int n) { // 求最后状态 dp[n] 的值
  if (n <= 1)
    return n;
  return fib(n - 1) + fib(n - 2);
}
// 2. 将递归转为动态规划(消除重复计算)
int fib(int n) {
  int[] dp = new int[3];
  dp[0] = 0; dp[1] = 1; // 初始记忆
  dp[2] = dp[0] + dp[1]; // dp[2] 是最新记忆
  for (int i = 2; i <= n; i++) {
    dp[0] = dp[1]; dp[1] = dp[2];// 更新记忆
    dp[2] = dp[0] + dp[1]; // 依赖前两个记忆
  }
  return dp[2];
}
\end{verbatim}

      \end{minipage}
    };
    %-----------------------------------------
    \node[fancytitle, right=10pt] at (box.north west) {动态规划:带记忆功能的(非)递归};
  \end{tikzpicture}



  %------------- START SECTION ---------------
  \begin{tikzpicture}
    \node [mybox] (box){%
      \begin{minipage}{0.3\textwidth}

        \begin{verbatim}
// 比如:模拟从黑箱子中取球的过程(有放回)
// 回溯不同于动态规划,动态规划有公式可循
// 用 arr 表示原始数组,用 used 剪枝优化
// 用 i == arr.length 判断递归是否终止
List<List<Integer>> ans = new ArrayList<>();
List<Integer> path = new ArrayList<>();
void dfs(int[] arr, boolean[] used, int i) {
  if (i == arr.length) {
    // 注意,深拷贝
    ans.add(new ArrayList<>(path));
    return;
  }
  // 每次都向 path 的第 j 个位置推送不同数字
  for (int j = 0; j < nums.length; j++) {
    if (!used[j]) {
      path.add(nums[j]);
      used[j] = true;
      dfs(nums, used, i + 1);
      used[j] = false; // 撤销原操作
      path.remove(path.size() - 1);
    }
  }
}
\end{verbatim}

      \end{minipage}
    };
    %-----------------------------------------
    \node[fancytitle, right=10pt] at (box.north west) {全排列问题:回溯、N 叉树遍历、穷举};
  \end{tikzpicture}



  %------------- START SECTION ---------------
  \begin{tikzpicture}
    \node [mybox] (box){%
      \begin{minipage}{0.3\textwidth}

        \begin{verbatim}
// "感染" 每个可连通的单元,由 1 变成 2
void infect(int[][] arr, int i, int j,
                         int N, int M) {
  if (i < 0 || i >= N || j < 0 || j >= M
            || arr[i][j] != 1)
    return;
  arr[i][j] = 2; // 感染
  infect(arr, i+1, j, N, M);
  infect(arr, i-1, j, N, M);
  infect(arr, i, j+1, N, M);
  infect(arr, i, j-1, N, M);
}
int count(int[][] arr) {
  if (arr == null || arr[0] == null)
    return 0;
  int N = arr.length;
  int M = arr[0].length;
  int ans = 0;
  for (int i = 0; i < N; i++)
    for (int j = 0; j < M; j++)
      if (arr[i][j] == 1) {
        ans++;
        infect(arr, i, j, N, M);
      }
}
\end{verbatim}

      \end{minipage}
    };
    %-----------------------------------------
    \node[fancytitle, right=10pt] at (box.north west) {岛问题:求连通区域的个数};
  \end{tikzpicture}



  %------------- START SECTION ---------------
  \begin{tikzpicture}
    \node [mybox] (box){%
      \begin{minipage}{0.3\textwidth}

        \begin{verbatim}
// next 数组记录最长相等的前后缀长度
void getNext(int[] next, String pat) {
  next[0] = 0;
  int j = 0; // 失配后的回退点
  // 循环从 1 开始,不是 0
  for (int i = 1; i < pat.length(); i++) {
    char chi = pat.charAt(i);
    char chj = pat.charAt(j);
    while (j > 0 && chi != chj)
      j = next[j - 1]; // 回退
    if (chi == chj)
      j++;
    next[i] = j;
  }
}
int strStr(String txt, String pat) {
  if (pat.length() == 0)
    return 0;
  int[] next = new int[pat.length()];
  getNext(next, pat);
  int j = 0;
  for (int i = 0; i < txt.length(); i++) {
    chi = txt.charAt(i);
    chj = pat.charAt(j);
    while (j > 0 && chi != chj)
      j = next[j - 1];
    if (chi == chj)
      j++;
    if (j == pat.length())
      return i - pat.length() + 1;
  }
  return -1;
}
\end{verbatim}

      \end{minipage}
    };
    %-----------------------------------------
    \node[fancytitle, right=10pt] at (box.north west) {KMP 算法:求 next 数组};
  \end{tikzpicture}



  %------------- START SECTION ---------------
  \begin{tikzpicture}
    \node [mybox] (box){%
      \begin{minipage}{0.3\textwidth}

        \begin{verbatim}
// 需要借助栈和指针来实现
void inorder(TreeNode root) {
  Stack<TreeNode> stack = new Stack<>();
  TreeNode cur = root;
  while (cur != null || !stack.isEmpty()) {
    if (cur != null) { // 遍历左子节点,入栈
      stack.push(cur);
      cur = cur.left;
    }
    else { // 遍历完左子节点,出栈,保存结果
      cur = stack.peek();
      stack.pop();
      // ans.add(cur.val);
      cur = cur.right;
    }
  }
}
\end{verbatim}

      \end{minipage}
    };
    %-----------------------------------------
    \node[fancytitle, right=10pt] at (box.north west) {中序遍历:迭代实现};
  \end{tikzpicture}



  %------------- START SECTION ---------------
  \begin{tikzpicture}
    \node [mybox] (box){%
      \begin{minipage}{0.3\textwidth}

        \begin{verbatim}
void preorder(TreeNode root) {
  Stack<TreeNode> stack = new Stack<>();
  if (root == null)
    return;
  stack.push(root);
  while (!stack.isEmpty()) {
    TreeNode cur = stack.peek();
    stack.pop();
    // ans.add(cur.val);
    if (cur.right != null) // 先右后左
      stack.push(cur.right);
    if (cur.left != null)
      stack.push(cur.left);
  }
}
\end{verbatim}

      \end{minipage}
    };
    %-----------------------------------------
    \node[fancytitle, right=10pt] at (box.north west) {前序遍历:迭代实现};
  \end{tikzpicture}



  %------------- START SECTION ---------------
  \begin{tikzpicture}
    \node [mybox] (box){%
      \begin{minipage}{0.3\textwidth}

        \begin{verbatim}
// 默认的初始化方法
PriorityQueue<Integer> pq = new PriorityQueue<>();

// 自定义排序规则
PriorityQueue<Integer> pq = new PriorityQueue<>(
  new Comparator<Integer>() {
    @Override
    public int compare(Integer o1, Integer o2) {
        return o1 - o2; // (升序) 谁小谁优先
    }
  }
);
\end{verbatim}

      \end{minipage}
    };
    %-----------------------------------------
    \node[fancytitle, right=10pt] at (box.north west) {优先级队列:默认是小根堆};
  \end{tikzpicture}



  %------------- START SECTION ---------------
  \begin{tikzpicture}
    \node [mybox] (box){%
      \begin{minipage}{0.3\textwidth}

        \begin{verbatim}
// 和前序遍历类似的代码,也需要借助栈来实现
// 遍历顺序不一样,且多了一个 reverse 环节
void postorder(TreeNode root) {
  Stack<TreeNode> stack = new Stack<>();
  TreeNode cur = root;
  if (root == null)
    return;
  stack.push(root);
  while (!stack.isEmpty()) {
    TreeNode cur = stack.peek();
    stack.pop();
    // ans.add(cur.val);
    if (cur.left != null) // 先左后右
      stack.push(cur.left);
    if (cur.right != null)
      stack.push(cur.right);
  }
  reverse(ans);
}
\end{verbatim}

      \end{minipage}
    };
    %-----------------------------------------
    \node[fancytitle, right=10pt] at (box.north west) {后序遍历:迭代实现};
  \end{tikzpicture}



  %------------- START SECTION ---------------
  \begin{tikzpicture}
    \node [mybox] (box){%
      \begin{minipage}{0.3\textwidth}

        \begin{verbatim}
reverseList(ListNode head) {
  if (head == null || head.next == null)
      return head;
  
  // 三指针翻转链表
  ListNode pre = null;
  ListNode cur = head;
  while (cur != null) {
    ListNode nxt = cur.next; // 临时保存
    cur.next = pre;
    // 更新节点
    pre = cur;
    cur = nxt;
  }
  return pre;
}
\end{verbatim}

      \end{minipage}
    };
    %-----------------------------------------
    \node[fancytitle, right=10pt] at (box.north west) {后序遍历:迭代实现};
  \end{tikzpicture}



  %------------ END OF CONTENT ---------------
\end{multicols*}

\end{document}
